%%%%%%%%%%%%%%%%%%%%%%%%%%%%%%%%%%%%%%%%%%%%%%%%%%%%%%%%%%%%%%%%%%%%%%%%%%%%%%%%
%                                                                              %
% Copyright (C) 2006-2014 Cassio Neri Moreira                                  %
%                                                                              %
% This work is licensed under a Creative Commons Attribution-ShareAlike 4.0    %
% International License.                                                       %
% http://creativecommons.org/licenses/by-sa/4.0/                               %
%                                                                              %
% First appeared in Portuguese:                                                %
% Cassio Neri, "O Problema Impossivel", Eureka!, 23, p 32--39, 2006.           %
% http://www.obm.org.br/opencms/revista_eureka/                                %
%                                                                              %
% The most updated version can be found at                                     %
% http://github.com/cassioneri/Impossible                                      %
%                                                                              %
%%%%%%%%%%%%%%%%%%%%%%%%%%%%%%%%%%%%%%%%%%%%%%%%%%%%%%%%%%%%%%%%%%%%%%%%%%%%%%%%

\documentclass[a4paper]{article}

\usepackage{amsmath, amsfonts, amssymb}
\usepackage[colorlinks=true,dvips]{hyperref}

\setlength{\hoffset}{0pt}
\setlength{\voffset}{0pt}
\setlength{\marginparsep}{0pt}
\setlength{\marginparwidth}{0pt}
\setlength{\evensidemargin}{0pt}
\setlength{\oddsidemargin}{0pt}
\setlength{\topmargin}{0pt}
\setlength{\textwidth}{\paperwidth}
\addtolength{\textwidth}{-2in}
\setlength{\textheight}{\paperheight}
\addtolength{\textheight}{-2in}
\addtolength{\textheight}{-\footskip}
\addtolength{\textheight}{-\headheight}
\addtolength{\textheight}{-\headsep}

\parskip.5\baselineskip

\title{The Impossible Problem\footnote{%
  First appeared in Portuguese: Cassio Neri, \href{http://www.obm.org.br/export/sites/default/revista_eureka/docs/eureka23.pdf}
  {``O Problema Impossivel''},
  \href{http://www.obm.org.br/opencms/revista_eureka/}{Eureka!}, 23, p32--39,
  2006.
  \newline
  Latest version available at
  \href{http://github.com/cassioneri/Impossible}
  {github.com/cassioneri/Impossible}.
  \newline
  This work is licensed under a
  \href{http://creativecommons.org/licenses/by-sa/4.0/}
  {Creative Commons Attribution-ShareAlike 4.0 International License}.
}}
\author{Cassio Neri}
\date{12 May 2014}

\begin{document}

\maketitle

%\begin{abstract}
%\end{abstract}

\maketitle

\section{Introduction}

\indent\vspace{-\baselineskip}

The Impossible Problem is a very beautiful problem about integer numbers. Its
original form was given by Freudenthal \cite{Freudenthal} before being
popularized by Martin Gardner \cite{Gardner}. The two problems are not exactly
the same. Gardner's version is as follows:

\begin{quotation}
Two numbers (not necessarily different) are chosen from the range of positive
integers greater than 1 and not greater than 20. Only the sum of the two numbers
is given to mathematician S. Only the product of the two is given to
mathematician P.

On the telephone S says to P, ``I see no way you can determine my sum.'' An hour
later P calls him back to say, ``I know your sum.'' Later S calls P again to
report, ``Now I know your product.''\\
What are the two numbers?
\end{quotation}

In the original problem Freudenthal gave an upper bound of $100$ (not for the
numbers but for their sum). To simplify the problem, Gardner gave the bound of
$20$. Doing that, ``the impossible problem turned out to be literally
impossible'' as said by Gardner himself \cite{Gardner2}. In this text we will
explain this point.

It's surprising that the original problem is well posed since one can think that
the little conversation has no relevant information about the numbers. But as we
shall see, this conversation is plenty of mathematical information.

The Impossible Problem is a $3$-in-$1$ problem. The problems for P and S differ
and none of them is the same as ours. Both P and S have additional information
(the product for P and the sum for S) that we don't. This makes a big
difference.

Section \ref{sec:Computer} shows how the problem can be solved with a computer's
help. Section \ref{sec:Mistake} explains why although the numbers are within
the range $[2, 20]$, if we set $N = 20$ then the solution no longer holds.
Section \ref{sec:PS} shows how P and S can solve their problems with no need of
a computer. Finally, Section \ref{sec:Hand} shows that we can also find the
solution without a computer.


\section{Solving the problem}
\label{sec:Computer}

\indent\vspace{-\baselineskip}

To solve the problem we shall use a computer program since there are too many
cases to consider. A code in C language, available on \cite{Code}, gives the
answer. Later, we shall see another solution that do not require computer
assistance.

Now we explain the method of solution. Roughly speaking, we begin with two
large sets: the set of all possible products and the set of all possible sums.
From each sentence of the conversation we extract information that reduces the
sizes of these sets. At the end, we have only one possible pair of product and
sum. By solving a second degree polynomial equation, we determine the numbers.

We assume the following fundamental hypothesis, otherwise the problem has no
mathematical meaning.

\noindent{\bf Fundamental Hypothesis}: P and S are telling the truth.

This hypothesis deserves a philosophical comment. A reader might think that such
hypothesis doesn't hold since S said that P couldn't find the sum and, an hour
later, P annouced that she did. Is anyone lying? Not necessarily. The hypothesis
is still plausible when time is considered. Initially, P couldn't solve the
problem. For some reason, that we shall see below, S knew that and told P. After
thinking, P figured out the reason why S was so sure. Only after that (hence,
not before S has spoken) P could determine the sum.

Let $p$ be the product and $s$ the sum of the two numbers. Let $N$ be the given
upper bound. In this section $N = 100$. We know {\it a priori} that $p$ is not
prime because it's a product of two integers greater than one. In addition, $p$
cannot have any prime divisor bigger than $N$ (for instance, $p$ cannot be
$2\cdot 101$). There are other restrictions, for instance, $p\ne 11\cdot 11\cdot
11$ and $p\ne 11\cdot 13\cdot 17$. Notice that we know a lot about $p$. On the
other hand, we have no restriction on $s$.

To simplify the presentation, we make the following definitions.

\noindent{\bf Definition}: Given an integer $m\ge 4$, we say that the pair
$(i,j)$ is a {\em factorization} of $m$ if $m=ij$ with $i$ and $j$ integers
in $[2, N]$.

\noindent{\bf Definition}: Given an integer $m\ge 4$, we say that the pair
$(i,j)$ is a {\em decomposition} of $m$ if $m=i+j$ with $i$ and $j$ integers
in $[2, N]$.

Notice that these definitions depends on $N$. So $(2,8)$ is a factorization of
$16$ if, and only if, $N\ge 8$. When refering to factorizations and
decompositions we make the convention that $(i,j)=(j,i)$.

The restrictions on $p$ seen above are consequences of the following fact: there
exist at least one factorization of $p$. The sets of admissible products and 
sums are given, respectively, by
\begin{align*}
P_0 & := \{ m \in\mathbb{N};\ 4\le m\le N^2 \text{ and $m$ has at least one
factorization} \},\\
S_0 & := \{ m \in\mathbb{N};\ 4\le m\le 2N \}.
\end{align*}

The program tels us that $P_0$ has $2880$ elements. It tels also that $S_0$ has
$197$ elements but this is trivially obtained with no need of a computer.

We now analyse the dialogue. First S says that P cannot determine the sum. This
means that $p$ has at least two distinct factorizations. Indeed, we already know
that $p$ has a factorization. If $(i, j)$ was the only one, then $P$ would know
that $s = i + j$. Hence, the set of possible products reduces to 
\[
P_1 := \{ m \in P_0\ ;\ m \text{ has at least two factorizations}\}.
\]
We have $p\in P_1$ and, accordingly to the program, $P_1$ has $1087$ elements.
One can easily see that if $i$ and $j$ are primes and $i,j\le N$, then $ij\in
P_0\setminus P_1$. However, numbers of this form constitute just a small part of
$P_0\setminus P_1$. Other examples of numbers in $P_0\setminus P_1$
are $N^2$ and $N(N-1)$.

There is another consequence of S's first sentence: for every decomposition $(i,
j)$ of $s$ we have $ij\in P_1$. Indeed, suppose that $s$ has a decomposition
$(i, j)$ such that $ij\notin P_1$, that is, $(i, j)$ is the unique factorization
of $ij$. At this stage, S couldn't deny the possibility of $p$ being $ij$ and,
if this was the case, then P would know that the numbers were $i$ and $j$. Hence
S wouldn't be sure that P was unable to find the sum. Therefore, the set
of admissible sums reduces to
\[
S_1 := \{ m \in S_0\ ; \text{ for each decomposition } (i,j) \text { of $m$ we 
have } ij\in P_1 \}.
\]
We have $s\in S_1$ and, accordingly to the program,
\[
S_1=\{11, 17, 23, 27, 29, 35, 37, 41, 47, 53\}.
\]

Next P says that she knows the sum. It follows that there exists a unique
factorization $(i, j)$ of $p$ such that $i + j\in S_1$. Indeed,
since $s\in S_1$, $p$ has a factorization (the one we are trying to find) whose
sum is in $S_1$. If there was another such factorization, then P could not know
which one was correct. Now, the set of admissible products reduces to
\[
P_2 := \{ m \in P_1\ ; \text{ there exists a unique factorization } (i,j)
\text{ of } m \text { with } i + j\in S_1 \}.
\]
We have $p\in P_2$. The computer program shows that this set has $86$ elements.

Finally, S clais that she knows the product. By an argument analougous to that of
the last paragraph, we conclude that there exists a unique decomposition
$(i, j)$ of $s$ such that $ij\in P_2$. Therefore, the set of admissible sums
shrinks to
\[
S_2 := \{ m \in S_1\ | \text{ there exists a unique decomposition } (i, j)
\text{ of } m \text{ with } ij\in P_2 \}.
\]
We have $s\in S_2$ and the program tells us that $S_2=\{17\}$, that is, $s=17$.

At this point S has announced having found the product. Since S is telling the
truth, we can find it too. Indeed, now we know that the sum is $17$. So we can
put ourselves in S's shoes. Another way to find the solution is reiterating the
step which gives $P_2$ from $S_1$ to construct $P_n$ from $S_{n-1}$ and that
which gives $S_2$ from $P_2$ to construct $S_n$ from $P_n$. We repeat until we
find $n\in \mathbb{N}$ such that the sets $P_n$ and $S_n$ are singletons. More
precisely, for $n\ge3$, we set
\begin{align*}
P_n &:= \{ m \in P_{n-1}\ | \text{ there exists a unique factorization } (i, j)
\text{ of } m \text{ with } i + j\in S_{n-1} \}. \\
S_n &:= \{ m \in S_{n-1}\ | \text{ there exists a unique decomposition } (i, j)
\text{ of } m \text{ with } ij\in P_n \}.
\end{align*}
We obtain $P_3=\{52\}$ and $S_3=\{17\}$, hence $p = 52$ and $s = 17$ and then
the numbers are $4$ and $13$.

\section{Gardner's mistake}
\label{sec:Mistake}

\indent\vspace{-\baselineskip}

Knowing the answer $4$ and $13$, intuition would say that the same solution
holds if we change, as Gardner did, the upper bound $N$ from $100$ to $20$.
Funilly enough, this is wrong! We shall see that for $N\le 61$ the numbers
$4$ and $13$ do not solve the problem! 

What happens if $N\le 61$? In this case $17\notin S_2$. In fact, the computer
program shows that $S_2=\varnothing$ which means the problem has no solution.
One can argue that this might be a default of the program and the solution
$4$ and $13$ can still hold. However, let us look at the mathematical argument
again. It's as follows:
\begin{enumerate}
\item Before the dialogue : $p\in P_0$ e $s\in S_0$;
\item After S's first sentence: $p\in P_1$ e $s\in S_1$;
\item After P's first sentence: $p\in P_2$;
\item After S's second sentence: $s\in S_2$.
\end{enumerate}

Therefore, if the numbers are $4$ and $13$, then $p = 52\in P_2\subset P_1
\subset P_0$ and $s = 17\in S_2\subset S_1\subset S_0$. Suppose now that $N\le
61$. We shall prove that $52\notin P_2$ or $17\notin S_2$ and it will follow
that the numbers cannot be $4$ and $13$. We proceed by contradiction assuming
that $52\in P_2$ and $17\in S_2$.

We claim that $19$, $37\notin S_1$. Indeed, since $34$ has at most one
factorization (which is $(2,17)$) we have $34\notin P_1$ and thus $19 = 2 + 17
\notin S_1$. Notice that $186$ can be written as a product of two natural
numbers only in the following ways: $1\cdot 186$, $2\cdot 93$, $3\cdot 62$ and
$6\cdot 31$. Since $N\le 61$ at most the last form is a factorization of $186$
and thus $186\notin P_1$. It follows that $37 = 6 + 31\notin S_1$.

Now, let us show that $70\in P_2$. We can write $70$ as $1\cdot 70$,
$2\cdot 35$, $5\cdot 14$ and $7\cdot 10$. As we have seen $37 = 2 + 35$ and
$19 = 5 + 14$ are not in $S_1$ but $17\in S_1$. Hence $70\in P_2$.

Finalmente, entre todas as decomposi��es de $17$ temos $(4,13)$ e $(7,10)$. Como
$52,70\in P_2$ temos que $17\notin S_2$ o que � uma contradi��o.

Finally, among all decompositions of $17$ we have $(4, 13)$ and $(7, 10)$. Since
$52$, $70\in P_2$ we have $17\notin S_2$, which is a contradiction.

I emphasize the following facts:
\begin{enumerate}

\item We have shown only that $(4, 13)$ is no longer a solution. Pehaps another
solution arises for $N\le 61$. But the program in \cite{Code} shows that,
in fact, $S_2 = \varnothing$ and thus the problem has no solution.

\item Even ther purists who don't accept computer assisted proofs should accept
that $(4, 13)$ is not a solution for $N\le 61$.

\item When I say the problem has no solution, I mean {\em our} problem has no
solution. On the other hand, the problem given to P have solution provided she
has good data. The situation for S is not very different from ours. We shall see
more details in the next section.
\end{enumerate}

\section{Problem for P and S}
\label{sec:PS}

\indent\vspace{-\baselineskip}

In this section we consider the other problems, specially, the one given to P.
In \cite{Code} we give a code in C language that could be used by P (or by you
in his place) and also a code for S. For the time being, we go back to assuming
that $N = 100$.

Despite the use of a computer to assist our solution, P and S don't need a great
computational ability to find the answer. A great memory (specially for S) or at
most a pencil and a piece of paper are enough for them.

P knows that $p = 52$ and its factorizations are $(2, 26)$ and $(4, 13)$. Thus
she knows that $s$ is either $28$ or $17$. After S talks for the first time, P
find the decompositions of $28$: $(2, 26)$, $(3, 25)$, $(4, 24)$, $(5, 23)$, ...
She can stop at $(5, 23)$ and conclude that $s\ne 28$ because $(5, 23)$ is the
unique factorization of $115$ ($5$ and $23$ are primes) and thus $115\notin
P_1$. She can now seek the decompositions of $17$ (just to be sure that $s =
17$): $(2, 15)$, $(3, 14)$, $(4, 13)$, $(5, 12)$, $(6, 11)$, $(7, 10)$ and $(8,
9)$. These pairs are factorizations of, respectively, $30$, $42$, $52$, $60$,
$66$, $70$ and, $72$. All these numbers are in $P_1$ since each of them has
another factorization which differ from the one shown here. Now P can conclude
that the sum of numbers are $17$ and thus she is able to find the two numbers.

S must work more than P. Nevertheless, she is also able to find the product
after some minutes%
%
\footnote{Playing the role of S, I found the product in about 15 minutes. To
simplify my task, in my computations I set $P_0\setminus P_1$ to have only
numbers which are product of two primes.}.
%

Now we consider other values of $N$. As we have seen, for smal values we cannot
solve the problem. Analogously, for $N$ large, {\em e.g.} $N = 866$, we are not
able to find the numbers since because $P_n = \{52, 244\}$ and $S_n=\{17,65\}$
for all $n\ge 3$. The effect of changing $N$ for P is almost null: knowing that
$p = 52$ she finds the solution $(4, 13)$ for all values of $N\ge 13$ (which
agrees with intuiton). Even the ambiguity that we have for $N = 866$ disappears.
In this case, had P been given $p = 244$ then she would find the solution
$(4, 61)$.

When $N\le 25$, $(2, 26)$ is no longer a factorization of $52$. In this case, P
would have no doubt that the $s = 17$. Hence S wouldn't say that P cannot
determine the sum. P could find $s$ but S wouldn't find $p$ as we shall see
below.

For $N\le 61$, S's situation is similar to ours: she cannot find $p$. The reason
is the same as before: $(4, 13)$ and $(7, 10)$ are two decompositions of $17$
with products in $P_2$. Another way to see this fact is the following. Let
$N = 61$ and put yourself in P's shoes, first, with $p = 52$ and, second, with
$p = 70$. In both cases, after S's first sentence, you would find $s = 17$.
Hence S couldn't decide whether $p = 52$ or $p = 70$. In fact there is a third
possibility for $N = 61$, namely, $p = 66$. For $N = 866$, S find the $p = 52$,
when $s = 17$, or $p = 244$, when $s=65$.

\section{Solving without a computer}
\label{sec:Hand}

\indent\vspace{-\baselineskip}

Let us see now another way to solve the original problem, where $N = 100$, that
doesn't need computer assistence.

Finding $P_0$ or $P_1$ manually would be very laborous (recall that, according
to the program, these sets have, respectively, $2880$ and $1087$ elements). Let
us find $S_1$ with no computer but a bit of patience.

To prove that $m\notin S_1$ it suffices to find a decomposition $(i, j)$ of $m$
which is the unique factorization of $ij$. Let's see some cases (whose
verification is left to the reader). For $m = 200$ we have the decomposition
$(100, 100)$. For $m = 199$ and $m = 198$ we have $(99, 100)$ and $(99, 99)$,
respectively.

For $99\le m\le 197$ we have $2\le m - 97\le 100$. Then $(i, j) = (97, m - 97)$
is a decomposition of $m$. Now, $97$ is a prime factor of $ij = 97(m - 97)$ and
then, either $i$ or $j$ is divisible by $97$. Assume, without loss of
generality, that $i$ is divisible by $97$. In this case, we must have $i = 97$
because, otherwise, $i$ has another factor and then $i\ge 2\cdot97>100$ which is
absurd. Hence $(97, m - 97)$ is the unique factorization of $97(m - 97)$. In the
same way, one can show that if $55\le m\le 153$, then $(53, m - 53)$ is the
unique factorization of $53(m - 53)$.

In summary, we have shown that if $m\in S_1$, then $m < 55$.

If $m\le 54$ is even, then one can verify, case-by-case, that $m$ has a
decomposition $(i, j)$ with $i$ and $j$ primes%
%
\footnote{The general case, {\em i.e}, with no upper bound for $m$, is an open
problem known as the Goldbach's conjecture \cite{Wikipedia}.}%
%
. Such decomposition is the unique factorization of $ij$. Analogously, for $m\in
\{5, 7, 9, 13, 15, 19, 21,$ $25, 31, 33, 39, 43, 45\}$, the decomposition $(i, j)=
(2, m - 2)$ has both $i$ and $j$ primes, meaning this is the only factorization
of $2(m-2)$. Finally, $(17, 34)$ is a decomposition of $51$ which is the unique
factorization of $17\cdot 34 = 2\cdot 17^2$ (since $17^2>100$).

It follows that $S_1\subset\tilde S_1 := \{11, 17, 23, 27, 29, 35, 37, 41, 47,
53\}$. Now, we shall prove the opposite inclusion to conclude that $S_1 = \tilde
S_1$. For this, we need to prove that for all $m\in\tilde S_1$ and every
decomposition $(i, j)$ of $m$ there exists a distinct factorization of $ij$.

Let $m\in\tilde S_1$ and $(i, j)$ be a decomposition of $m$. Since $m$ is odd,
either $i$ of $j$ is even. Without loss of generality, we assume $i$ is even.
We consider first the case $i \ge 4$ and then the case $i = 2$.

If $i\ge 4$, then $j = m - i \le m - 4\le 53 - 4 < 50$ and we obtain $2j < 100$.
Hence, $(i/2, 2j)$ is a factorization of $ij$ which distinct from $(i, j)$
unless $i = 2j$ and $m = 3j$. The unique multiple of $3$ in $\tilde S_1$ is $27$
for which the decomposition in question is $(18, 9)$ but its product also admits
the factorization $(6, 27)$.

If $i = 2$, then we have to present a factorization of $2(m - 2)$ distinct from
$(2, m - 2)$, which is done in Table \ref{tab:S1}.

\begin{table}[ht]
\center
\begin{tabular}{cccc}
\hline
$m$  & $(2, m - 2)$ & $2(m - 2)$ & Distinct decomposition \\
\hline
$11$ & $(2, 9)$     & $18$       & $(3, 6)$ \\
$17$ & $(2, 15)$    & $30$       & $(3, 10)$ \\
$23$ & $(2, 21)$    & $42$       & $(3, 14)$ \\
$27$ & $(2, 25)$    & $50$       & $(5, 10)$ \\
$29$ & $(2, 27)$    & $54$       & $(3, 18)$ \\
$35$ & $(2, 33)$    & $66$       & $(3, 22)$ \\
$37$ & $(2, 35)$    & $70$       & $(5, 14)$ \\
$41$ & $(2, 39)$    & $78$       & $(3, 26)$ \\
$47$ & $(2, 45)$    & $90$       & $(3, 30)$ \\
$53$ & $(2, 51)$    & $102$      & $(3, 34)$ \\
\hline
\end{tabular}
\caption{Completing the proof that $\tilde S_1\subset S_1$.}
\label{tab:S1}
\end{table}

Finding $P_2$ would also be very labourous and we go directly to $S_2$. By
definition $m\in S_2$ if there exists a unique decomposition $(i, j)$ of $m$
which is also the unique factorization of $ij$ whose sum belongs to $S_1$.
Hence, to conclude that $m\in S_1\setminus S_2$ it suffices to find two
distinct decompositions of $m$ that are the unique factorizations of their
product having sum in $S_1$. This is done for all $m\in S_1\setminus\{17\}$ in
Table \ref{tab:S2}. It follows that $S_1\setminus\{17\}\subset S_1\setminus S_2$
or, in other words, $S_2\subset\{17\}$. Since $s\in S_2$ we have $S_2=\{17\}$
and $s = 17$.

\begin{table}[ht]
\center
\footnotesize
\begin{tabular}{c|cccc|cccc}
\hline
$m$  & 1$^{st}$ decomp. & Product & Other fact(s).
     & Sum(s)
     & 2$^{nd}$ decomp. & Product & Other fact(s).
     & Sum(s) \\
\hline
$11$ & $(2, 9)$         & $18$    & $(3, 6)$    
     & $9$
     & $(4, 7)$         & $28$    & $(2, 14)$
     & $16$ \\
$23$ & $(4, 19)$        & $76$    & $(2, 38)$
     & $40$
     & $(10, 13)$       & $130$   & $(2, 65)$, $(5, 26)$
     & $67$, $31$ \\
$27$ & $(2, 25)$        & $50$    & $(5, 10)$
     & $15$
     & $(4, 23)$        & $92$    & $(2, 46)$
     & $48$ \\
$29$ & $(4, 25)$        & $100$   & $(2, 50)$, $(5, 20)$
     & $52$, $25$
     & $(6, 23)$        & $138$   & $(2, 69)$, $(3, 46)$
     & $71$, $49$ \\
$35$ & $(4, 31)$        & $124$   & $(2, 62)$
     & $64$
     & $(6, 29)$        & $174$   & $(2, 87)$, $(3, 58)$
     & $89$, $61$ \\    
$37$ & $(6, 31)$        & $186$   & $(2, 93)$, $(3, 62)$
     & $95$, $65$
     & $(8, 29)$        & $232$   & $(4, 58)$
     & $62$ \\
$41$ & $(6, 35)$        & $210$   & $(3, 70)$
     & $73$
     & $(7, 34)$        & $238$   & $(14, 17)$
     & $31$ \\
$47$ & $(4, 43)$        & $172$   & $(2, 86)$
     & $88$
     & $(6, 41)$        & $246$   & $(3, 82)$
     & $85$ \\
$53$ & $(4, 49)$        & $196$   & $(2, 98)$
     & $100$
     & $(10, 43)$       & $430$   & $(5, 86)$
     & $91$ \\
\hline
\end{tabular}
\caption{The proof that $S_2\subset\{17\}$. Notice that no number in either
{\em Sum(s)}' columns are in $S_1$.}
\label{tab:S2}
\end{table}

We know that $p = ij$ for some decomposition $(i, j)$ of $17$. In addition, $p
\in P_2$, that is, $(i, j)$ is the unique factorization of $p$ whose sum is in
$S_1$. We shall show now that $(4, 13)$ is the only decomposition of $17$ with
this property and it will follow that the numbers are $4$ and $13$.

Indeed $52$ has only two factorizations $(2, 26)$, $(4, 13)$ but only the latter
has the sum in $S_1$. It remains to prove that every other decomposition
$(i, j)$ of $17$ has a product that admits a factorization distinct from
$(i, j)$ with sum in $S_1$. This is done in Table \ref{tab:17}.

\begin{table}[ht]
\center
\begin{tabular}{cccc}
\hline
Decomposition  & Product & Factorization & Sum \\
\hline
$(2, 15)$      & $30$    & $(5, 6)$  & $11$ \\
$(3, 14)$      & $42$    & $(2, 21)$ & $23$ \\
$(5, 12)$      & $60$    & $(3, 20)$ & $23$ \\
$(6, 11)$      & $66$    & $(2, 33)$ & $35$ \\
$(7, 10)$      & $70$    & $(2, 35)$ & $37$ \\
$(8, 9)$       & $72$    & $(3,24)$  & $27$ \\
\hline
\end{tabular}
\caption{Any decomposition of $17$ but $(4, 13)$ have a product that doesn't
belong to $P_2$.}
\label{tab:17}
\end{table}

\begin{thebibliography}{00}

\bibitem{Freudenthal}
{\sc Hans Freudenthal,}
Nieuw Archief Voor Wiskunde, Ser 3, 17 (1969) 152.

\bibitem{Gardner}
{\sc Martin Gardner,}
Mathematical Games, Scientific American, 241 (Dec. 1979) 22.

\bibitem{Gardner2}
{\sc Martin Gardner,}
Mathematical Games, Scientific American, 242 (May 1980).

\bibitem{Wikipedia}
{\sc Wikepedia,}
{\em Goldbach's conjecture}, \url{http://en.wikipedia.org/wiki/Goldbach\%27s_conjecture}.

\bibitem{Sallows}
{\sc Lee Sallows,}
The Impossible Problem, The Mathematical Intelligencer, 17:1 (1995) 27.

\item{Isaacs}
{\sc I.M. Isaacs,}
The Impossible Problem Revisited Again, The Mathematical Intelligencer, 17:4
(1995) 4.

\bibitem{Code} URL: \verb+www.labma.ufrj.br/~cassio/f-impossible.html+

\end{thebibliography}

\end{document}
